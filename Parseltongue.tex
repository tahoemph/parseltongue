\documentclass[]{article}
\usepackage{minted}
\usepackage[T1]{fontenc}
\usepackage{lmodern}
\usepackage{amssymb,amsmath}
\usepackage{ifxetex,ifluatex}
\usepackage{fixltx2e} % provides \textsubscript
% use upquote if available, for straight quotes in verbatim environments
\IfFileExists{upquote.sty}{\usepackage{upquote}}{}
\ifnum 0\ifxetex 1\fi\ifluatex 1\fi=0 % if pdftex
  \usepackage[utf8]{inputenc}
\else % if luatex or xelatex
  \ifxetex
    \usepackage{mathspec}
    \usepackage{xltxtra,xunicode}
  \else
    \usepackage{fontspec}
  \fi
  \defaultfontfeatures{Mapping=tex-text,Scale=MatchLowercase}
  \newcommand{\euro}{€}
\fi
% use microtype if available
\IfFileExists{microtype.sty}{\usepackage{microtype}}{}
\ifxetex
  \usepackage[setpagesize=false, % page size defined by xetex
              unicode=false, % unicode breaks when used with xetex
              xetex]{hyperref}
\else
  \usepackage[unicode=true]{hyperref}
\fi
\hypersetup{breaklinks=true,
            bookmarks=true,
            pdfauthor={},
            pdftitle={Parseltongue},
            colorlinks=true,
            citecolor=blue,
            urlcolor=blue,
            linkcolor=magenta,
            pdfborder={0 0 0}}
\urlstyle{same}  % don't use monospace font for urls
\setlength{\parindent}{0pt}
\setlength{\parskip}{6pt plus 2pt minus 1pt}
\setlength{\emergencystretch}{3em}  % prevent overfull lines
\setcounter{secnumdepth}{0}

\title{So you wanna speak Parseltongue?}
\author{}
\date{}

\begin{document}
\maketitle

\tableofcontents

\section{Introduction}

We all learned a first real language. For me I started with BASIC on a
Vic 20 in 1981 or so. Later Forth and 6502. But the first real
language was c. It still is the mother tongue. It is the best for
highly performant software and is contains the most reused libraries.
But for many tasks there are languages that are better suited from a
software engineering standpoint. In this document I'll describe steps
you can take to learn to write in Python without too much of an accent.
By learning how to write Python in a relatively native manner you make
it easier for others to read your code, make it more likely you will use
constructs that perform well, and will make it easier on yourself when
you need to read others code.

This document will assume you have read a basic tutorial but are now
trying to fit the pieces together.

\section{History}

Python was conceived and originally developed by Guido Van Rossum as a
replacement for the teaching language ABC starting in the late 80s.
From the standpoint of writing modern Python the interesting part of
the history begins in 2008 when Python 3 is released. Since then the
Python 2.x line is capped at 2.7. Until fairly recently most of the
major Python libraries have not been available in Python 3.

Python is managed by the Python Foundation. There are a series of
proposals for extending the language and the libraries called PEPs.

\section{Categorization}

Python is a multi-paradigm language. Its basic control flow structures
are like those of other Algol based languages. Its module and object
oriented structures are adapted from modula. And it has many functional
features taken from haskell and other languages.

\section{Stylistic structure}

Guido set the pace for stylistic structure fairly early. PEP 8 lays out
what good source should look like. Unlike many early languages this is
fairly well followed. Some parts of the standard library were developed
in various styles, but in general the PEP 8 style pervades the
community.

There is a general zen of python that you can read by typing ``import
this'' at a python REPL prompt. Fundamentally it encourages you to
stick to things that are easy to read and simple to comprehend.

\section{Interesting Features}

\subsection{List Comprehensions}

This is a concept taken from functional languages. As noted later one
of the best way to get good performance from python is to spend as much
time in code written in c as possible. So when building a data
structure expressing that in a comprehension will produce the best
performance by far.

So don't do:

\begin{minted} {python}
values = []
for i in range(10):
    values.append(i**2)
\end{minted}

Instead do:

\begin{minted} {python}
values = [i**2 for i in range(10)]
\end{minted}

\subsection{Generators}

Initially generators just look like a cute way to hide iteration over a
data structure. But more fundamentally they provide a performant way to
build pipelines of data. When I was first learning python I was
building a ETL system for financial data. One (well, many) of the
vendors regularly provided what should have been data in a CSV format
but it contained an extra quotation mark if the last entry was a string.

So something like:

``company 1'', ``address'', ``05112014'', 500.0, ``barf''''

It would have been easy to just fix this with a script that passed over
the data generating a valid file but it sucks to start running over the
data twice. So instead it is easy to build a generator to fix this
problem. Like:

\begin{minted} {python}
def fix_busted_data(source):
    for line in source:
    yield line[:-1] # simplification
\end{minted}

Now it was easy to do something like:

\begin{minted} {python}
with open(``sourcedata.csv'', ``r'') as f:
    for line in fix_bused_data(f):
        process_line(line)
\end{minted}

You can build much deeper hierarchies of generators pulling data through
a series of filters. Note that generators in the c version of python
are on the stack of the caller and are much faster than a normal
function call.

David Beazley has presented on generators (and a related concept
coroutines) a fair number of times including a recent python meetup in
SF. His talks at
\href{http://www.google.com/url?q=http\%3A\%2F\%2Fwww.dabeaz.com\%2Fgenerators\%2F\&sa=D\&sntz=1\&usg=AFQjCNHgQSwBOSX7OQcp3SSqls_7gn5JwQ}{http://www.dabeaz.com/generators/}are
worthwhile to work your way through.

\subsection{Object Oriented Features}

Python provides a normal set of OO features. The multiple inheritance
features are simplified to some extent and in the code I usually see
rarely used. Since python 2 the type system is unified such that even
the builtin types act in a normal fashion.

\subsection{Functional Features}

Over time many functional features have been added to python. In
addition to the list comprehensions above these include first class
functions. You often see this feature used when metaprogramming with
decorators.

\subsection{Meta Programming}

\section{Performance considerations}

Stay off the grass. I mean the interpreter. If you can express
something in a way that

\section{Test Frameworks}

\section{Logging}

\section{GIL}

GIL stands for ``Global Interpreter Lock''. When in the interpreter
python takes a big lock. That makes it single threaded. This means
that it is hard to achieve parallelism using core python features. So
you achieve parallelism by starting up other processes.

Concurrency on the other hand can be easily achieved in python.

\section{Development Environments}

Python is a scripting languages. A core piece of your development
environment is the REPL. When you run across something you are
uncertain about you should start here. Any good development environment
will start here.

There is a debugger in the core of Python. Two thing that are useful
are the ability to embed breakpoints into code and the ability to run
your code under the debugger from its initialization. These look like:

\begin{minted}{python}
if possible_complex_scenario(inputs):
    import pdb; pdb.set_trace()
\end{minted}

When you hit this condition this will stop you at this point and drop
you into the debugger.

\begin{minted}{bash}
$ python -mpdb nesoi.py
> /opt/nesoi-work/nesoi/nesoi.py(1)<module>()
-> import argparse
(Pdb) b 76
Breakpoint 1 at /opt/nesoi-work/nesoi/nesoi.py:76
(Pdb) b report
Breakpoint 2 at /opt/nesoi-work/nesoi/nesoi.py:147
(Pdb) c
Traceback (most recent call last):
  File "/usr/local/lib/python2.7/pdb.py", line 1314, in main
    pdb._runscript(mainpyfile)
  File "/usr/local/lib/python2.7/pdb.py", line 1233, in _runscript
    self.run(statement)
  File "/usr/local/lib/python2.7/bdb.py", line 400, in run
    exec cmd in globals, locals
  File "<string>", line 1, in <module>
  File "nesoi.py", line 11, in <module>
    some_bug()
NameError: name 'some_bug' is not defined
Uncaught exception. Entering post mortem debugging
Running 'cont' or 'step' will restart the program
> /opt/nesoi-work/nesoi/nesoi.py(11)<module>()
-> some_bug()
(Pdb) 
\end{minted}

There are several things being demonstrated here. The first is how to
start up a script with a module loaded and initialized. In this case it
is the debugger package and you are stopped at the first line. Next you
can set breakpoints. Last if you hit an exception you are stopped,
given a stack trace, and able to look around before restarting. This
provides for a powerful and tight debug loop.

I use vim with a couple of extensions. Chief among these are
scrooloose/syntastic and klen/python-mode. This, coupled with tmux,
makes for a very good environment for programmers who work on the home
row.

I have used JetBrains PyCharm in the past. I think this is a pretty
good environment if you are not home row oriented. I gave up on it a
few years ago because of its mouse orientation getting in the way of
writing code and its speed. I've heard its performance has been
significantly enhanced since. There is a community and commercial
version of this package so you can try out its basic functionality
inexpensively.

For doing interactive analysis people should look at IPython. This is
an awesome environment built around applying code cells to data and
producing visuals in a notebook style of development. This was built by
a Physicist originally at CU Boulder and later at Berkeley. It has many
features which fit together in a really holistic manner. Combined with
numpy, scipy, and pandas this is the core for a nice python analysis
environment.

\section{Other Resources}

The way you learn how to program in any language is to read source
written by people who are good at it. For example nobody learns c well
without reading unix kernel source. In the case of Python reading the
standard library source is a good way to learn.

\end{document}
